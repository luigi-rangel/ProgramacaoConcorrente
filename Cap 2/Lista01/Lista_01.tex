\documentclass[a4paper, 12pt]{article}
\usepackage{amsmath}
\usepackage{amssymb}
\usepackage{amsthm}
\usepackage{mathabx}
\usepackage{geometry} 
\usepackage{pgfplots}
\pgfplotsset{compat=1.15}
\usepackage{mathrsfs}
\usetikzlibrary{arrows}
\usepackage{indentfirst}
\usepackage{enumerate}
\usepackage{enumitem}
\usepackage{listings}
\usepackage[breaklinks]{hyperref}
\usepackage{hyphenat}
\usepackage{subcaption}

\geometry{left=2.5cm,right=2.5cm,top=2.5cm,bottom=2.5cm}
\newtheorem{question}{Questão}
\newtheorem{questioni}{$\,\,$}[question]
\newtheoremstyle{dotless}{}{}{\itshape}{}{\bfseries}{}{ }{}
\theoremstyle{dotless}
\newtheorem*{answer}{\textbf{Resposta:}}
\renewcommand*{\thequestioni}{\alph{questioni}}

\lstdefinestyle{CStyle}{
    commentstyle=\color{green},
    keywordstyle=\color{magenta},
    numberstyle=\tiny\color{gray},
    stringstyle=\color{purple},
    basicstyle=\footnotesize,
    breakatwhitespace=false,         
    breaklines=true,                 
    captionpos=b,                    
    keepspaces=true,                 
    numbers=left,                    
    numbersep=5pt,                  
    showspaces=false,                
    showstringspaces=false,
    showtabs=false,                  
    tabsize=2,
    language=C
}

\title{2023.2 Programação Concorrente}
\author{Luigi Rangel, DRE: 121077746}
\date{27/08/2023}

\begin{document}

\maketitle

\begin{center}
    \line(1,0){400}
\end{center} \newpage

\begin{question}
    No programa de multiplicação de matrizes mostramos (Cap 2 - texto) uma forma de paralelizar o algoritmo de multiplicação de matrizes criando um fluxo de execução independente para calcular cada um dos elementos da matriz de saída. Proponha outra solução onde a tarefa de cada fluxo de execução seja calcular uma linha inteira da matriz de saída.
\end{question}
\begin{answer}
\end{answer}

\begin{lstlisting}[style=CStyle]
    #define N 1000 //N igual a dimensao da matriz

    float a[N][N], b[N][N], c[N][N];

    void calculaLinhaMatriz(int dim, int i) {
        int j, k, soma;
        for(j = 0; j < dim; j++) {
            soma = 0;

            for(k = 0; k < dim; k++) {
                soma += a[i][k] * b[k][j];
            }

            c[i][j] = soma;
        }
    }
    void main() {
        int i;
        //inicializa as matrizes a e b (...)
        //faz C = A * B
        for(i = 0; i < N; i++) {
            //dispara um fluxo de execucao f para executar:
            //calculaLinhaMatriz(N, i);
        }
    }
\end{lstlisting}

\newpage

\begin{question}
    Para arquiteturas de hardware com poucas unidades de processamento (como é o caso das CPUs multiicores) geralmente é melhor criar uma quantidade de fluxos de execução igual ao número de unidades de processamento. Altere a solução do exercício anterior fixando o numero de fluxos de execução e dividindo o cálculo das linhas da matriz de saída entre eles.
\end{question}
\begin{answer}
\end{answer}

\begin{lstlisting}[style=CStyle]
    #define N 1000 //N igual a dimensao da matriz
    #define P 4 //P igual ao numero de unidades de processamento

    float a[N][N], b[N][N], c[N][N];

    void calculaSegmentoMatriz(int dim, int ini, int fim) {
        int i, j, k, soma;
        for (i = ini; i < fim; i++) {
            for(j = 0; j < dim; j++) {
                soma = 0;

                for(k = 0; k < dim; k++) {
                    soma += a[i][k] * b[k][j];
                }

                c[i][j] = soma;
            }
        }
    }
    void main() {
        int i, inicio, fim;
        //inicializa as matrizes a e b (...)
        //faz C = A * B
        for(i = 0; i < P; i++) {
            inicio = (N * i) / P;
            fim = N * (i + 1) / P;
            //dispara um fluxo de execucao f para executar:
            //calculaSegmentoMatriz(N, inicio, fim);
        }
    }
\end{lstlisting}

\newpage

\begin{question}
    A série mostrada abaixo pode ser usada para estimar o valor da constante $\pi$. A função \texttt{piSequencial()} implementa o calculo dessa série de forma sequencial. Proponha um algoritmo concorrente para resolver esse problema dividindo a tarefa de estimar o valor de $\pi$ entre M fluxos de execução independentes.

    \[\pi = 4(1 - \frac{1}{3} + \frac{1}{5} - \frac{1}{7} + \frac{1}{9} - \dots)\]

    \begin{lstlisting}[style=CStyle]        
        double piSequencial (long long n) {
            double soma = 0.0, fator = 1.0;
            long long i;
            for (i = 0; i < n; i++) {
                    soma = soma + fator/(2*i+1);
                    fator = -fator;
                }
            return 4.0 * soma;
        }
    \end{lstlisting}
\end{question}
\begin{answer}
\end{answer}

\begin{lstlisting}[style=CStyle]
    #define N 100 //N igual a quantidade de termos da serie a serem calculados
    #define P 4 //P igual a quantidade de unidades de processamento

    void calculaSegmentoPi(double *seg, int ini, int fim) {
        int i, fator = 1 - 2 * (ini % 2);
        double soma = 0;
        
        for (i = ini; i < fim; i++) {
            soma += fator / (2 * i + 1);
            fator *= -1;
        }

        *seg = soma;
    }

    void main() {
        int i, inicio, fim;
        double pi = 0;
        double *segs;
        //aloca P espacos de double na memoria para a variavel segmentos

        for (i = 0; i < P; i++) {
            inicio = N * i / P;
            fim = N * (i + 1) / P;
            //dispara um fluxo de execucao f para executar
            //calculaSegmentoPi(&segs[i], inicio, fim);
        }

        //espera o fim de todos os fluxos de execucao

        for (i = 0; i < P; i++) {
            pi += segs[i];
        }
        pi *= 4;
    }
\end{lstlisting}

\newpage

\begin{question}
    A série infinita mostrada abaixo estima o valor de $log(1+x)$ $(-1 < x <=1)$:

    \[\log{(1 + x)} = x - \frac{x^2}{2} + \frac{x^3}{3} - \frac{x^4}{4} + \frac{x^5}{5} - \dots\]

    Dois programas foram implementados para calcular o valor dessa série (um programa sequencial e outro concorrente) usando N termos. Apos a implementação, foram realizadas execuções dos dois programas, obtendo as medidas de tempo apresentadas na Tabela 1. A coluna N informa o numero de elementos da série, a coluna thread informa o número de threads, e as colunas $T_s$ e $T_c$ informam os tempos de execução do programa sequencial e do programa concorrrente, respectivamente.

    \begin{table}[h!]
        \centering
        \begin{tabular}{|c|l|l|l|l|}
            \hline
            N               & threads & Ts (s) & Tc (s) & A \\
            \hline
            $1 \times 10^6$ & 1       & 0,88   & 0,89   &   \\
            \hline
            $1 \times 10^6$ & 2       & 0,88   & 0,50   &   \\
            \hline
            $1 \times 10^7$ & 1       & 8,11   & 8,34   &   \\
            \hline
            $1 \times 10^7$ & 2       & 8,11   & 4,44   &   \\
            \hline
            $2 \times 10^7$ & 1       & 16,21  & 16,41  &   \\
            \hline
            $2 \times 10^7$ & 2       & 16,21  & 8,84   &   \\
            \hline
        \end{tabular}
        \caption{Medidas de tempo para calcular $\log{(1+x)}$.}
    \end{table}
\end{question}

\begin{questioni}
    Complete a coluna A com os valores de aceleração.
\end{questioni}
\begin{answer}
\end{answer}

\begin{table}[h!]
    \centering
    \begin{tabular}{|c|l|l|l|l|}
        \hline
        N               & threads & Ts (s) & Tc (s) & A    \\
        \hline
        $1 \times 10^6$ & 1       & 0,88   & 0,89   & 0,99 \\
        \hline
        $1 \times 10^6$ & 2       & 0,88   & 0,50   & 1,76 \\
        \hline
        $1 \times 10^7$ & 1       & 8,11   & 8,34   & 0,97 \\
        \hline
        $1 \times 10^7$ & 2       & 8,11   & 4,44   & 1,82 \\
        \hline
        $2 \times 10^7$ & 1       & 16,21  & 16,41  & 0,99 \\
        \hline
        $2 \times 10^7$ & 2       & 16,21  & 8,84   & 1,83 \\
        \hline
    \end{tabular}
    \caption{Aceleração preenchida.}
\end{table}

\vspace*{0.5cm}

\begin{questioni}
    Avalie os resultados obtidos para essa métrica. Considere os casos em que a carga de dados aumenta junto com o número de processadores e os casos isolados onde apenas a carga de trabalho ou o número de processadores aumenta.
\end{questioni}
\begin{answer}
\end{answer}

Quando o número de processadores aumenta, mas a carga de trabalho não, há aceleração sublinear. Entretanto, quando a carga de trabalho aumenta mas o número de processadores não, a aceleração permanece quase igual. Ou seja, o programa paralelo se mostra útil neste caso para ser implementado em arquiteturas com 2 ou mais processadores.

\newpage

\begin{question}
    Considere uma aplicação na qual $20\%$ do tempo total de execução é comprometido com tarefas sequenciais e o restante, $80\%$, pode ser executado de forma concorrente.
\end{question}

\begin{questioni}
    Se dispusermos de uma máquina com 4 processadores, qual será a aceleração teórica (de acordo com a lei de Amdahl) que poderá ser alcançada em uma versão concorrente da aplicação?
\end{questioni}
\begin{answer}
\end{answer}

A aceleração teórica será de $\frac{1}{0.2 + (0.8 / 4)} = 2.5$.

\vspace*{0.5cm}

\begin{questioni}
    Se apenas $50\%$ das atividades pudessem ser executadas em paralelo, qual seria a aceleração teórica considerando novamente uma máquina com 4 processadores?
\end{questioni}
\begin{answer}
\end{answer}

A aceleração teórica seria de $\frac{1}{0.5 + (0.5 / 4)} = 1.6$.

\begin{center}
    \line(1,0){400}
\end{center}

\end{document}